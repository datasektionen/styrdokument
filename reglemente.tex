\documentclass{dgovdoc}

\usepackage[swedish]{babel}
\usepackage[T1]{fontenc}
\usepackage[utf8]{inputenc}

\usepackage{hyperref}

\title{Reglemente}

\begin{document}

\maketitle

\section{Insignia}

\subsection{Färg}

Sektionens färg är cerise.

\subsection{Symbol}

Sektionens symbol är den grekiska bokstaven lilla delta. Deltat får icke
glömmas. Utformningen och användandet av sektionens symbol regleras närmare av
Informationsorganet i samråd med D-rektoratet.

\subsection{Grafisk profil}

Sektionens grafiska profil regleras av Informationsorganet i samråd med
D-rektoratet.

\section{Andra styrande dokument}

Sektionen har utöver stadgar och reglemente även ett ekonomiskt styrdokument, samt
ett antal policydokument. Sektionens policydokument består av dFunkpolicyn, jämlikhetspolicyn,
alkoholpolicyn, rekryteringspolicyn och informationsspridningspolicyn.
Dessa dokument reglerar sektionens verksamhet och kompletterar stadgar och reglemente.
Beslut om ändringar i dessa dokument tas på SM.

\section{Nämnder}

Det åligger samtliga nämnder, funktionärer, samt projektledare att följa de dokument
som tas upp i §2. D-rektoratet ansvarar för att samtliga nämnder har en kontaktperson
i D-rektoratet, samt att nämndordförande vet vem detta är.

\subsection{Sektionslokalsgruppen}

\subsubsection{Ändamål}

Sektionslokalsgruppens syfte är att sköta, underhålla och utveckla
sektionslokalen.

\subsubsection{Organisation}

Sektionslokalsgruppen kallas för METAdorerna och leds av sektionslokalsansvariga, en från Sektionen för
Medieteknik och en från Konglig Datasektionen, som väljs av respektive sektion.
Funktionären från Datateknik kallas för Konglig Lokalchef. Beslut som fattas av de sektionslokalsansvariga måste vara enhälligt, och i de
fall som sektionslokalsansvariga är oense åligger det respektive
sektionsordföranden att komma överens om och besluta i frågan.
Sektionslokalsansvariga utser resten av gruppen efter ansökningar och efter att
de ansökande en gång ha lagat middag åt nämnden, sig själva inkluderade.
Sektionslokalsgruppen bestämmer internt organisation och planering.

\subsubsection{Verksamhet}

Sektionslokalsgruppen ansvarar för:

\begin{itemize}
  \item administration av bokning och uthyrning av sektionslokalen samt dess
    olika rum
\end{itemize}

\begin{itemize}
  \item tekniken, möbler/inredning, samt påfyllning av förbrukningsmaterial i
    sektionslokalen
\end{itemize}

\begin{itemize}
  \item utveckling av sektionslokalen
\end{itemize}

\begin{itemize}
  \item organisering av såväl måndagsstäd som annan städning av
sektionslokalen
\end{itemize}

\begin{itemize}
  \item att upprätta och underhålla ett dokument med gällande regler för
    sektionslokalen.
\end{itemize}


\subsection{Idrottsnämnden}

\subsubsection{Ändamål}

Idrottsnämnden skall verka för att sektionsmedlemmarna får tillfälle att
idrotta tillsammans.

\subsubsection{Organisation}

Idrottsnämnden leds av Sektionsidrottsledaren. Övriga medlemmar är samtliga
intresserade sektionsmedlemmar.

\subsubsection{Verksamhet}

Nämndens verksamhet planeras i början av varje verksamhetsår av
Sektionsidrottsledaren i samråd med medlemmarna.

\subsection{Informationsorganet}

\subsubsection{Ändamål}

Informationsorganet, Ior, ansvarar för informationsspridning på sektionen och
skall verka för att förbättra informationsflödet till, från och mellan
sektionens medlemmar.

\subsubsection{Organisation}

Ior leds av Kommunikatören, internt benämnd Chefsåsnan. Övriga medlemmar
är samtliga intresserade sektionsmedlemmar.

\paragraph{Undergrupper}

Informationsorganets undergrupper är delmängder av Informationsorganet
med särskilda ansvar och uppgifter. Undergrupperna förfogar
över sina eventuella respektive delar av informationsorganets budget
och leds av sina respektive funktionärer i samråd med Kommunikatören.
Samtliga undergrupper delar informationsorganets
övergripande ansvar och uppgifter samt ska vara Informationsorganet
och andra nämnder behjälpliga inom sina särskilda arbetsområden.

\subparagraph{Crash \& Bränn}

Crash \& Bränn ansvarar för alla sektionens datasystem, inklusive servrar som drivs för sektionens räkning. De leds av Systemansvarig.

\subparagraph{Tag Monkeys}

Tag Monkeys ansvarar för sektionens grafiska utveckling och
arbete. Leds av Datas Art Director.

\subparagraph{Redaqtionen}

Redaqtionen ansvarar för skriverier, nyhetsutskick och att ge
ut sektionstidningen dBuggen. Redaqtionen ska specifikt uppdatera
och ge ut exakt en n0lledBuggen till varje mottagning.
Priset för ett exemplar av dBuggen eller n0lledBuggen skall
vara noll prisbasbelopp. Redaqtionen leds av Chefredaqtören,
internt benämnd CheFred.

\subsubsection{Verksamhet}

Ior skall

\begin{itemize}
  \item Hjälpa D-rektoratet och övriga nämnder med informationsspridning
  \item Underhålla och utveckla sektionens webbplats
  \item Ansvara för drift och underhåll av sektionens datorresurser.
\end{itemize}

\subsection{Jämlikhetsnämnden}

\subsubsection{Ändamål}

Jämlikhetsnämnden skall värna och upplysa om jämlikhet och mångfald på
sektionen.

\subsubsection{Organisation}

Nämnden leds av Jämlikhetsnämndens ordförande. Styrelseledamoten för
studiesociala frågor skall också ingå i nämnden tillsammans med övriga
intresserade sektionsmedlemmar.

\subsubsection{Verksamhet}

Jämlikhetsnämnden skall

\begin{itemize}
  \item göra sektionsmedlemmarna medvetna om vad de har för rättigheter och
    vart de skall vända sig om de känner sig kränkta eller trakasserade
  \item arbeta för att öka jämlikheten och mångfalden på sektionen, i dess
    nämnder och funktionärsposter
  \item hålla minst två möten per termin där aktuella frågor gällande jämlikhet
    diskuteras
  \item skapa och kontinuerligt uppdatera en jämlikhetspolicy för sektionen
  \item assistera D-rektoratet med att ta fram, följa upp och vid behov
    revidera en handlingsplan mot trakasserier
  \item ha löpande samarbete med KTH, CSC och THS om jämlikhetsarbeten.
\end{itemize}

\subsection{Klubbmästeriet}

\subsubsection{Ändamål}

Klubbmästeriet, DKM, anordnar fester och andra sociala arrangemang för
sektionens medlemmar och i vissa fall även deras eventuella vänner.

\subsubsection{Organisation}

DKM leds av klubbmästaren. Övriga medlemmar utses av DKM.

\subsubsection{Verksamhet}

Det åligger DKM att

\begin{itemize}
  \item arrangera fester, jippon, högtidliga ceremonier och andra sociala
    arrangemang
  \item vid behov assistera vid andra sektionsrelaterade arrangemang
  \item skicka ut inbjudningar till övriga sektioner och andra högskolor till
    av DKM arrangerade evenemang
  \item i sektionens informationskanaler informera om till klubbmästeriet
    inkomna inbjudningar till externa evenemang, samt fördela platser och
    biljetter om antalet aspiranter överstiger platsantalet
  \item skicka ut skriftlig inbjudan till Vårbalen till medlemmar av sektionens
    ordnar.
\end{itemize}

\subsubsection{Bokföringsplikt}

DKM är bokföringspliktigt.

\subsection{Mottagningen}

\subsubsection{Ändamål}

Mottagningen har som syfte att ta hand om och roa de nyantagna på till
sektionen hörande program, främst på grundnivå men i tillämplig utsträckning
även på avancerad nivå, och få dem att lära känna varandra och äldre
sektionsmedlemmar. Vidare syftar Mottagningen till att lära nØllan hur KTH, THS
och sektionen fungerar och är uppbyggda utifrån ett studentperspektiv.
Mottagningen skall även sträva efter att ge nØllan bästa möjliga introduktion
till deras studier, till sektionen och till studentliv i allmänhet.

\subsubsection{Organisation}

Mottagningen som helhet leds gemensamt av Konglig Öfverdrif och Storasyskon, hädanefter 
benämnda Presidiet. Presidiet ansvarar inför D-rektoratet för Mottagningens verksamhet och ekonomi.

Mottagningen består av fem huvudsakliga grenar: Det Kongliga Dadderiet, Det Kongliga Drifveriet, Det Kongliga Doqumenteriet, Det Kongliga Mammeriet och Det Kongliga Ekonomeriet.

Presidiet väljer ut resten av mottagningens ledningsgrupp, Titel. Presidiet är en del av Titel och ansvariga för att sammankalla och leda Titels arbete. Titel arbetar både strategiskt och operativt. Under mottagningens gång fungerar Titel som gruppchefer för respektive gren.

Mottagningens sammansättning beslutas gemensamt av hela Titel.

\subsubsection{Rekrytering till Mottagningen}

Titelgruppen skall utlysa platserna i höstens mottagning under våren. Alla
sökande skall erbjudas intervjuer innan mottagningens sammansättning fastslås.
De sökandes egna prioriteringar av sökta uppdrag inom Mottagningen skall
respekteras.

\subsubsection{Sektionsordförandes roll}

Sektionsordförande är genom KTH:s och THS regler för Mottagningen ytterst
ansvarig för denna. Presidiet skall därför löpande hålla sektionsordförande
informerad om verksamheten och samråda med denne i frågor av principiell vikt.
Sektionsordförande skall å sin sida fungera som stöd åt presidiet och bistå dem
i deras ledningsfunktion.

Sektionsordförande äger alltid rätt att fatta de beslut och vidta de åtgärder
som denne finner vara nödvändiga för att KTH:s och THS regler för Mottagningen
skall upprätthållas. I den mån det är möjligt skall dock samråd alltid ske med
presidiet inför ett sådant beslut eller åtgärd.

Sektionsordförande bör inte inneha något annat uppdrag inom Mottagningen än det
som följer av ordförandeskapet.

\subsubsection{Verksamhet}

Mottagningen ansvarar för att aktiviteter som främjar Mottagningens ändamål
anordnas. Dessa aktiviteter avslutas sedan med nØlleGasquen och nØllans
eventuella upphöjelse till Ettan. För planeringen av verksamheten ansvarar
Titelgruppen.

\subsubsection{Bokföringsplikt}

Mottagningen är bokföringspliktig.

\subsection{Näringslivsgruppen}

\subsubsection{Ändamål}

Näringslivsgruppen har till uppgift att informera näringslivet om sektionen och
datateknikprogrammet, att främja sektionsmedlemmarnas status på arbetsmarknaden
samt att inbringa sponsorpengar till sektionen.

\subsubsection{Organisation}

Näringslivsgruppen leds av Näringslivsansvarig. Inom näringslivsgruppen är D-Dagenansvarig
ytterst ansvarig för sektionens arbetsmarknadsdag. Övriga medlemmar är samtliga in-
tresserade sektionsmedlemmar.

\subsubsection{Verksamhet}

Näringslivsgruppen skall

\begin{itemize}
  \item göra reklam för sektionen och datateknikprogrammet
  \item samordna de av sektionens verksamheter som riktar sig mot näringslivet,
    så att företagen bemöts på ett professionellt sätt
  \item hålla och ständigt förbättra kontakten med näringslivet
  \item sköta Näringslivsgruppen faktureringar
  \item se till att sektionen uppfyller avtal framförhandlade av
    Näringslivsgruppen.
\end{itemize}

\subsubsection{Avtal}

Näringslivsansvarig och D-Dagenansvarig har rätt att förhandla fram avtal med företag
och organisationer. Firmatecknare har rätt att utfärda en fullmakt till Näringslivsan-
svarig samt D-Dagenansvarig för att dessa ska kunna ingå i sagda avtal

\subsubsection{Bokföringsplikt}
Näringslivsgruppen är bokföringspliktig.

\subsection{Qulturnämnden}

\subsubsection{Ändamål}

Qulturnämnden skall verka för att anordna sociala evenemang för intresserade medlemmar ifrån Datasektionen och Mediasektionen. Detta kan innefatta allt från qulturella aktiviteter till aktiviteter som enligt normen klassas som nördiga. QN skall även förse sektionen med sällskapsspel samt ett rikt kulturutbud för sektionens medlemmar att kunna ta del av.

\subsubsection{Organisation}

Qulturnämnden benämns vanligtvis QN. QN leds av Qulturattachén tillsammans med motsvarande ansvarig funktionär ifrån Mediasektionen. QN har inget medlemsskap utan alla intresserade är välkomna att delta.

\subsubsection{Verksamhet}

QN bör
\begin{itemize}
  \item anordna regelbundna aktiviteter som höjer den kulturella nivån på sektionen, såsom filmvisningar, spelkvällar och andra underhållande aktiviteter.
  \item tillhandahålla spel och annat kulturellt i sektionslokalen. Detta kan innefatta böcker, uråldriga tv-spel samt brädspel som inte går att finna längre.
  \item årligen utse och Q-märka en person, ett ting, en företeelse eller något annat som man anser är ett gott exempel på god qultur. Q-märkningen skall lämpligen förevigas i form av ett tygmärke."
\end{itemize}

\subsubsection{Budget}
Qulturnämnden har en gemensam budget, som fastställs av respektive sektionsmöte. Kostnader och intäkter delas ej lika mellan sektionerna utan ansvarig funktionär från varje sektion avgör själv, eller i samråd med ansvarig funktionär ifrån den andra sektionen, hur dennes sektions del av budgeten ska användas.

\subsection{Studienämnden}

\subsubsection{Ändamål}

Studienämndens syfte är att bevaka och förbättra utbildningskvaliteten och
studiemiljön för sektionens medlemmar på kort såväl som lång sikt.

\subsubsection{Organisation}

Studienämnden leds av Studienämndens ordförande. Övriga medlemmar är
\begin{itemize}
	\item Programansvarig student
	\item Studerandeskyddsombud
	\item Ledamot för utbildningsfrågor
	\item Årskursrepresentanter för de tre första årskurserna
	\item Kursansvariga studenter för alla aktuella kurser
	\item Övriga intresserade THS-medlemmar
\end{itemize}

\paragraph{Årskursrepresentant}

Årskursrepresentanter väljs av Studienämndens ordförande varje läsår. Årskursrepresentanterna
är årskursernas representanter och kontakter i Studienämnden. Årskursrepresentanterna
ansvarar för att det finns Kursansvarig student för kurserna årskursen
läser och ska samordna och stötta dem i sitt arbete.

\paragraph{Kursansvarig student}

På varje kurs som ingår i sektionens utbildning ska det finnas en från Studienämnden
utnämnd Kursansvarig student (KAS). KAS är en student som går kursen och väljs för
en kursomgång i samband med kursstart.

KAS ska fungera som kontakten mellan Studienämnden, Kursansvarig och studenterna
som läser kursen och därigenom framföra synpunkter och lösa problem som kan dyka
upp under kursens gång. Under kursen ska KAS dokumentera synpunkter och kritik för
uppföljning på länkmöten och framtida kursomgångar samt bilda sig en uppfattning om
hur kursen sett ut tidigare och uppfölja om eventuell tidigare kritik på kursen åtgärdats.


\subsubsection{Verksamhet}

Studienämnden skall hålla möte minst en gång per månad under terminstid. Mötena
skall vara öppna för alla THS-medlemmar. Den huvudsakliga verksamheten skall vara
att
\begin{itemize}
	\item både proaktivt och reaktivt inhämta studenternas åsikter om studiesituationen och med dessa som utgångspunkt arbeta för att förbättra kvaliteten på utbildningen
	\item utvärdera information och beslut från KTH:s organ
	\item bevaka och söka förbättra den fysiska och psykosociala studiemiljön
\end{itemize}

Studienämndens dokument skall i så stor utsträckning som möjligt finnas tillgängliga i
elektronisk form.

\subsection{Konglig Östrogennämnden}

\subsubsection{Ändamål}

Nämndens syfte är att främja tjejers intressen på Datasektionen.

\subsubsection{Organisation}

Ordförande för Konglig Östrogennämnden är Öfvermatronan, som väljs på SM.

\subsubsection{Verksamhet}

Konglig Östrogennämnden ska verka för att ge tjejer på datasektionen en
möjlighet att nätverka med varandra. Nämnden ska anordna middag för alla tjejer
två gånger per termin samt anordna en tjejgasque en gång per år. Nämnden ska
medverka under mottagningen för att få fler tjejer att känna sig välkomna och
även anordna företags/inspirations-events under året. Nämnden ska varje år
dessutom utse en hedersdam.

\subsection{DEMON - Datas Eminenta MusikOrganisationsNämnd}

\subsubsection{Ändamål}

Nämndens syfte är att främja musikintresset på sektionen.

\subsubsection{Organisation}

Ordförande för DEMON är ÄrkeDEMON som väljs på Val-SM, har läsår som
mandatperiod. Övriga medlemmar är intresserade sektionsmedlemmar.

\subsubsection{Verksamhet}

DEMON verkar för att främja musikintresset på sektionen genom att anordna
regelbunda träffar där man bland annat kan få:

\begin{itemize}
\item Repa i sektionslokalen.
\item Dela med sig av musik.
\item Gå på DJ-kurs.
\item Umgås med likasinnade människor.
\end{itemize}

DEMON förespråkar även att man tillsammans ska gå på
konserter/festivaler/spelningar.

\subsection{Valberedningen}

\subsubsection{Ändamål}

Valberedningen har till uppgift att opartiskt administrera och bereda de
val som genomförs vid sektionen utan att göra bedömningar av kandidaters
lämplighet. Valberedningens Ordförande ansvarar för att Valberedningens
uppdrag utförs i enlighet med sektionens och THS stadgar samt reglemente.
Det är lämpligt att Valberedningens Ordförande nomineras till
THS Kårfullmäktiges valnämnd.

\subsubsection{Organisation}

Valberedningen leds av Valberedningens Ordförande. Valberedningen består av
Ordförande som valts på SM samt de ledamöter som utses av DM. Det bör
sammanlagt vara mellan 3 till 9 ledamöter. Samtliga sektionsmedlemmar
har rätt att bli nominerade och kandidera till ledamot i Valberedningen.

\subsubsection{Verksamhet}

Valberedningen ska uppmuntra till sektionsengagemang och bistå med information
och vägledning om detta bland sektionens medlemmar. Valberedningen ska utlysa
samtliga val och erbjuda samtliga sektionsmedlemmar möjlighet att nominera och
nomineras till de poster som utlysts. En kandidat är en nominerad sektionsmedlem.
Om en kandidat tackar ja skall Valberedningen informera denne om valprocessen.

Alla de utlåtanden som Valberedningen erhåller från de kandidater som valt att
fortsätta med sin kandidatur, antingen i form av en intervju eller frågeformulär,
skall samtidigt göras tillgängliga för sektionens medlemmar enligt sektionens
informationsspridningspolicy i god tid, dock senast en dag innan sektionsmötet
där valet avgörs.

\subsubsection{Utlåtande}

Det bör hållas intervjuer med samtliga kandidater som sökt till D-rektoratet,
Revisorer, Näringslivsansvarig, D-Dagenansvarig, Storasyskon, Konglig Öfverdrif samt Klubbmästare.
Det bör vara 2 personer närvarade vid en intervju med en kandidat. Det ska föras
anteckningar under intervjun. Med hjälp av det material som insamlats under
intervjun skall Valberedningen formulera ett skriftligt, objektivt utlåtande om
kandidatens kandidatur, dock ej bedöma kandidatens lämplighet. Detta skall innehålla:

\begin{itemize}
\item Redogörelse för kandidatens egenskaper i relation till posten ifråga.
\item Tidigare erfarenheter som bedöms vara av relevans till posten ifråga.
\item Annan information som bedöms vara relevant för SM:s beslutsfattande.
\end{itemize}

När detta utlåtandet är formulerat skall den kandidat som utlåtandet gäller
få ta del av utlåtandet och tillfrågas om godkännande av publicering.
Utlåtandet får inte publiceras utan detta godkännande.

I de fall där kandidaten inte intervjuas skall Valberedningen tillhandahålla
ett frågeformulär och resultatet av detta skall ses som Valberedningens utlåtande.

\subsubsection{Sekretess}

Allt intervjumaterial, exempelvis anteckningarna från dessa, samt de interna
diskussioner som Valberedningen har inför formulerandet av skriftliga utlåtanden
skall beläggas med sekretess. Endast valberedare och revisorer får närvara vid
dessa interna diskussioner. Detta är till för att skydda de som är med i
valberedningen, de kandidater som diskuteras och valprocessen som helhet.
Det innebär att det som är sekretessbelagt kommer hållas hemligt från alla,
i all framtid, utom de närvarande och revisorerna. Ingen annan får ta del av detta,
exempelvis D-Rektoratet och framtida valberedare.

\subsubsection{Urnval}

Vid urnval ska valberedningen hålla valperiod under minst fem läsdagar direkt
innan sektionsmötet. Resultatet av urnvalet skall redovisas på sektionsmötet
där valet hålls och där protokollföras. Urnval kan hållas digitalt varpå det
ska övervakas av sektionens revisorer med hjälp av systemansvarig. Vid icke-digitalt
urnval ska valurnan hållas tillgänglig för sektionens medlemmar i sektionslokalen
eller annan likvärdig minst en timme per dag, i första hand under lunchtid.
Dessa tider ska senast fem dagar i förväg annonseras enligt sektionens
informationsspridningspolicy.

Vid urnval används alternativomröstning. Väljarna rankar kandidaterna, där
vakans och blankröst skall vara alternativ som kan rankas. Om ingen av
kandidaterna uppnår enkel majoritet blir bottenkandidaten eliminerad, dock
ej vakans, och dennes röster blir omfördelade till väljarnas nästahandsval av
kandidat. Denna process upprepas tills dess att en kandidat uppnår enkel
majoritet, varpå valet skall godkännas av SM enligt stadgans §1.7.

\subsection{Datasektionens E-sports Community}

\subsubsection{Ändamål}

Datasektionens E-Sports Community, DESC, ansvarar för att stilla sektionens
medlemmars törst efter ett rikt utbud av e-sportsevenemang.

\subsubsection{Organisation}

DESC leds av Desctopen. Övriga medlemmar är intresserade
sektionsmedlemmar.

\subsubsection{Verksamhet}

DESC bör:

\begin{itemize}
\item Anordna ett flertal e-sportsevenemang per kalenderår, såsom hemmakommenterade
barcrafts, tävlingsLAN eller nybörjarevenemang.
\item Tillhandahålla en godtycklig mängd spelarkonton i ledande e-sporter.
\item Samarbeta med utomstående företag och/eller organisationer i syfte att sektionens
medlemmar ska få utöva e-sport på en högre nivå.
\item Skapa en tillhörighet i sektionens e-sportsutövande så att även nybörjare vill engagera
sig.
\end{itemize}

\subsection{Prylmångleriet}

\subsubsection{Ändamål}

Prylmångleriets ändamål är att förse sektionens medlemmar med balla prylar, till exempel märken, spegater, sångböcker och profilkläder.

\subsubsection{Organisation}
Prylmångleriet leds av Prylmånglaren. Övriga medlemmar utses av Prylmånglaren från intresserade sektionsmedlemmar.

\subsubsection{Verksamhet}
Prylmångleriet skall kontinuerligt se till att det finns prylar och dylikt till hands. Om något skulle ta slut skall det, om efterfrågan finns, beställas nya.

Prylmångleriet skall regelbundet ordna tillfällen där sektionens medlemmar kan köpa prylar. Sådana tillfällen skall ordnas inför större fester och evenemang.

Prylmångleriet skall under Mottagningen göra sig synlig bland nØllan och arrangera tillfällen då nØllan får prova overaller. Prylmånglaren har som ansvar att överse införskaffandet av ettans overaller. Prylmångleriet skall också hjälpa ettan att utforma och beställa årskursmärken.

Vid jubileum och andra större händelser på sektionen bör Prylmångleriet i samarbete med ansvariga för händelsen utforma och beställa prylar relaterade till händelsen.

\subsubsection{Bokföringsplikt}
Prylmångleriet är bokföringspliktigt.

\section{Funktionärer}

\subsection{D-rektoratet}

\subsubsection{Sektionsordförande}

Är ledamot i sektionsstyrelsen, D-rektoratet. Arbetsleder D-rektoratet och företräder
organisationen utåt. Är firmatecknare tillsammans med kassören. Har det övergripande
ansvaret för sektionens avtalshantering och serveringstillstånd. Ansvarar även för att
det upprättas en verksamhetsberättelse varje år som talar om vad som hänt under året.
Dessa ansvarsområden inkluderar att:
\begin{itemize}
	\item förmedla kontakt utifrån till delar av organisationen och verka för givande samarbeten
	\item tillsammans med lokalchefen ansvara för sektionslokalen gentemot KTH
	\item förvalta sektionens externa relationer
	\item fånga upp ansvar som inte direkt faller på någon annan funktionär på sektionen
	\item stödja kassören i det ekonomiska arbetet och ha en god överblick av sektionens ekonomi
\end{itemize}

Väljs på Glögg-SM. Har kalenderår som mandatperiod.

\subsubsection{Vice sektionsordförande}

Är ledamot i sektionsstyrelsen, D-rektoratet. Har kontakten med funktionärerna som
huvudsakligt ansvarsområde. Fungerar som stöd för resterande styrelseledamöter och
arbetar speciellt nära ordförande. Dessa ansvarsområden inkluderar att:
\begin{itemize}
	\item arbetsleda ledamöterna i deras arbete och kommunikation med funktionärerna
	\item delegera och följa upp på beslutsuppföljning inom styrelsen
	\item kunna företräda organisationen tillsammans med ordförande
	\item se efter avtal som upprättas inom sektionen, såsom nyckelavtal och accesser
\end{itemize}

Väljs på Glögg-SM. Har kalenderår som mandatperiod.

\subsubsection{Sekreterare}

Är ledamot i sektionsstyrelsen, D-rektoratet. Sekreteraren arbetar med formalia och
styrdokument som huvudsakliga ansvarsområden. I detta ingår även arbete med kommunikation
och kunskapsbevarande gällande dessa gentemot alla sektionens medlemmar.
Dessa ansvarsområden inkluderar att:
\begin{itemize}
	\item protokoll från DM och SM anslås i enlighet med stadgarna
	\item stötta funktionärer och övriga medlemmar i sitt arbete med styrdokumenten
	\item tillsammans med styrelsen bibehålla styrdokument samt övrig formalia i gott skick
	\item hålla sektionens posthantering fungerande
	\item verka för goda överlämningsrutiner i sektionens verksamhet
	\item tidigt vara med i arbetet att ta fram utkast till verksamhetsplan och verksamhetsberättelse för sektionen
\end{itemize}

Väljs på Glögg-SM. Har kalenderår som mandatperiod.

\subsubsection{Kassör}

Är ledamot i sektionsstyrelsen, D-rektoratet. Är firmatecknare tillsammans med ordfö-
rande och har det övergripande ansvaret för sektionens ekonomi. Detta innebär ansvar
för budget och sektionens likvida medel, främst bankkonton och handkassor. Dessa
ansvarsområden inkluderar även att:
\begin{itemize}
	\item arbeta strategiskt med sektionens ekonomiska frågor
	\item arbetsleda övriga bokföringspliktiga nämnder i bokföringsarbetet
	\item fungera som ett stöd för sektionens funktionärer i ekonomiska frågor
	\item ansvara för att upprätta och följa upp budgeten
	\item arbeta strategiskt för att uppnå de ekonomiska målen satta av sektionen
	\item ansvarig för sektionens bokföring
	\item se efter att rutinerna för sektionens likvida medel följs
\end{itemize}

Väljs på Glögg-SM. Har kalenderår som mandatperiod.

\subsubsection{Ledamot för näringsliv och kommunikation}

Är ledamot i sektionsstyrelsen, D-rektoratet. Arbetar med och för i styrelsen de frågor
som rör näringsliv och kommunikation, samt är kontaktperson för sektionens verksamhet
som avser dessa områden. Dessa ansvarsområden inkluderar att:
\begin{itemize}
  \item överblicka och verka för stärkandet av varumärket Datasektionen
  \item strategiskt arbeta med utvecklingen av sektionens olika verktyg för intern och extern kommunikation tillsammans med berörda funktionärer
  \item strategiskt arbeta med utvecklingen av sektionens näringslivsverksamhet
\item bistå Näringslivsgruppen i samordnandet av näringslivsfrågor på sektionen
\end{itemize}

Väljs på Val-SM. Har läsår som mandatperiod.

\subsubsection{Ledamot för studiesociala frågor}

Är ledamot i sektionsstyrelsen, D-rektoratet. Har frågor som rör medlemmarnas psykiska
och fysiska miljö som ansvarsområde. Arbetar med att utveckla medlemmarnas
studentliv. Dessa ansvarsområden inkluderar att:

\begin{itemize}
  \item strategiskt arbeta med utvecklingen av sektionens studiesociala verksamhet
  \item verka för en god studiemiljö
  \item upprätthålla och förvalta kontakten med andra sektioner och organisationer
  \item verka för en god sammanhållning mellan sektionens engagerade medlemmar
  \item från styrelsen arbeta med frågor som berör jämlikhet, mångfald och likabehandling på sektionen
\end{itemize}

Väljs på Val-SM. Har läsår som mandatperiod.

\subsubsection{Ledamot för utbildningsfrågor}

Är ledamot i sektionsstyrelsen, D-rektoratet. Har det övergripande ansvaret för sektionens
studentinflytande gentemot CSC och för sektionens utbildningspåverkan. Är
styrelsens kontakt för frågor som rör dessa områden.
Dessa ansvarsområden inkluderar att:
\begin{itemize}
	\item verka för att upprätthålla en god kontakt med ansvariga på andra sektioner om hur dessa arbetar med studentinflytande och utbildningspåverkan.
	\item ha en samordnade roll för de som arbetar med studentinflytande gentemot CSC på sektionen.
	\item strategiskt arbeta med utvecklingen av sektionens utbildningspåverkan.
	\item från styrelsen samordna sektionen i frågor som rör utbildningspåverkan
\end{itemize}

Väljs på Val-SM. Har läsår som mandatperiod.

\subsection{Nämndordförande}

\subsubsection{Kommunikatör}

Är ordförande för Informationsorganet.

Väljs på Val-SM. Har läsår som mandatperiod.

\subsubsection{Jämlikhetsnämndens ordförande}

Är ordförande för Jämlikhetsnämnden.

Väljs på Val-SM. Har läsår som mandatperiod.

\subsubsection{Klubbmästare}

Är ordförande för Klubbmästeriet.

Väljs på Val-SM. Har läsår som mandatperiod.

\subsubsection{Konglig Lokalchef}

Konglig lokalchef är sektionslokalsansvarig och leder sektionslokalsgruppen
tillsammmans med motsvarande post vid Sektionen för Medieteknik.

Väljs på Val-SM. Har läsår som mandatperiod.

\subsubsection{Konglig Öfverdrif}

Är tillsammans med Storasyskon ansvarig för Mottagningen.

Väljs på Glögg-SM. Har kalenderår som mandatperiod.

\subsubsection{Näringslivsansvarig}

Är ordförande för Näringslivsgruppen. Ansvarar för kontakter med näringslivet. I händelsen att posten som D-Dagenansvarig är vakantsatt ansvarar Näringslivsansvarig även för D-dagen tills dess att posten som D-Dagenansvarig kan fyllnadsväljas.


Väljs på Val-SM. Har läsår som mandatperiod.

\subsubsection{Qulturattaché}

Är ordförande för Qulturnämnden.

Väljs på Val-SM. Har läsår som mandatperiod.

\subsubsection{Sektionsidrottsledare}

Är ordförande för Idrottsnämnden.

Väljs på Val-SM. Har läsår som mandatperiod.

\subsubsection{Storasyskon}

Är tillsammans med Konglig Öfverdrif ansvarig för Mottagningen.

Väljs på Glögg-SM. Har kalenderår som mandatperiod.

\subsubsection{Studienämndens ordförande}

Är ordförande för Studienämnden.
Arbetsleder Studienämnden och är sammankallande för dess möten. Har
det övergripande ansvaret för nämndens verksamhet. Dessa ansvarsområden inkluderar
att:
\begin{itemize}
	\item verka för att alla sektionsmedlemmar är representerade i Studienämnden
	\item synpunkter och förslag på förbättringar från sektionsmedlemmarna samlas in och förs fram till skolan och kursansvariga i de kurser som medlemmarna läser
	\item verka för att alla sektionsmedlemmar känner till Studienämnden, dess syfte och hur man kan påverka sin utbildning genom den
	\item studienämndens arbete bevaras och dokumenteras för uppföljning av framtida medlemmar
\end{itemize}

Väljs på Val-SM. Har läsår som mandatperiod.

\subsubsection{Öfvermatrona}

Är ordförande för Konglig Östrogennämnden.

Väljs på Val-SM. Har läsår som mandatperiod.

\subsubsection{ÄrkeDEMON}

Är ordförande för DEMON.

Väljs på Val-SM. Har läsår som mandatperiod.

\subsubsection{Valberedningens ordförande}

Är ordförande för valberedningen.

Väljs på Val-SM. Har läsår som mandatperiod.

\subsubsection{Desctop}

Är ordförande för Datasektionens E-Sports Community.

Väljs på Val-SM. Har läsår som mandatperiod.

\subsubsection{Prylmånglaren}
Är ordförande för Prylmångleriet.

Väljs på Glögg-SM. Har kalenderår som mandatperiod.

\subsection{Övriga funktionärer}

\subsubsection{Fanbärare}

Fanbärarna försvarar sektionens ära genom att bära dess fana vid olika
högtidliga tillfällen. Observera att fanan skall hållas högt.

Att vara Fanbärare är en mycket hedersfylld post på sektionen.

Fanbärarna skall närvara på så många som möjligt av de tillställningar, till
vilka de inbjuds av THS, samt i andra sammanhang efter beslut av D-rektoratet.

Sektionen bekostar fanbärarnas alkoholfria deltagaravgift för de evenemang där
fanan bärs.

Fanbärarna bär huvudansvaret för att sektionens fana hålls i gott skick.

Väljs på Glögg-SM. Har kalenderår som mandatperiod.

\subsubsection{Vice fanbärare}

Vice fanbärare försvarar sektionens ära när ordinarie fanbärare ej har
möjlighet att göra det. Vid arrangemang med begränsat deltagarantal har
fanbäraren företräde framför vice fanbäraren.

Väljs på Glögg-SM. Har kalenderår som mandatperiod.

\subsubsection{Kårfullmäktigeledamöter}

\paragraph{Ändamål}

Kårfullmäktigeledamöterna och -suppleanterna representerar sina väljare i
THS Kårfullmäktige. Sektionen fungerar som en valkrets och såväl ordinarie ledamöter
som suppleanter är valda på personligt mandat vid THS. Dessa antas även som funktionärer
vid Datasektionen i syfte att föra THS Kårfullmäktiges arbete närmare den
egna valkretsen

\paragraph{Organisation}

Sektionen har en ordinarie ledamot och en suppleant för varje mandat i THS Kårfullmäktige
som sektionen tilldelats i enlighet med THS styrdokument. Mandatperioden för samtliga
ordinarie ledamöter och suppleanter regleras i THS styrdokument.

\paragraph{Verksamhet}

Såväl Kårfullmäktigeledamöter som -suppleanter skall delta på så många sammanträden
av THS Kårfullmäktige som möjligt. De är solidariskt ansvariga för att sektionen
är fulltalig vid samtliga Kårfullmäktigesammanträden. En ledamot eller injusterad suppleant
har inget ansvar att rösta i sektionens intresse, utan ska rösta så som den själv
finner lämpligast. I egenskap av funktionärer har dessa även ett ansvar att förmedla
information från THS Kårfullmäktige till sektionens medlemmar.

\subsubsection{Programansvarig student}

Är ansvarig för sektionens
utbildningspåverkan på programnivå och arbetar från stundentsidan med utbildningsprogrammets
utformning. Är ansvarig för kontakten gentemot CSC-skolan samt KTH
i frågor gällande den långsiktiga utvecklingen av programmet och dess kurser i sin helhet.
Bör ha ha god kunskap om programmets uppbyggnad och kurserna som ingår.

Väljs på Glögg-SM. Har kalenderår som mandatperiod.

\subsubsection{Revisorer}

\paragraph{Ändamål}

Revisorernas uppgift är att övervaka D-rektoratet och nämndernas arbete.

\paragraph{Organisation}

Enligt sektionens stadgar finns två revisorer, utsedda av SM. De skall

\begin{itemize}
  \item övervaka D-rektoratet och nämndernas arbete i sektionens namn,
  \item övervaka den löpande bokföringen och, om så anses behövas, kräva att en
    delårsrapport presenteras,
  \item revidera ekonomisk bokföring från sektionens organ,
  \item övervaka upprättandet av verksamhetsberättelsen för sektionen, samt
  \item vara skiljemän vid tvister inom sektionen där parterna inte behöver
    använda sig av SM eller THS styrelse, revisorsgrupp eller Kårfullmäktige.
\end{itemize}

Tvister där sektionens revisorer inte kan vara skiljemän inkluderar, men är
inte begränsat till, tvister där revisorerna kan anses jäviga.

\paragraph{Verksamhet}

Revisorerna för ett verksamhetsår är ålagda att revidera samtliga av sektionens
verksamheter för det året, samt att i samråd med tidigare och senare revisorer
revidera löpande verksamhet som löper över flera år. Det åligger de senast
valda revisorerna att ansvara för att revisionerna genomförs.

\subparagraph{Revisionsberättelse och -rapport}

Varje revision dokumenteras i två skrivelser. Av dessa två är
revisionsberättelsen offentlig.

Revisionsrapporten är en detaljerad beskrivning av anmärkningar i bokföring
och/eller verksamhet. Den ligger till grund för kommunikationen mellan
sektionens revisorer från år till år. I
revisionsrapporten bör antecknas

\begin{itemize}
  \item anmärkningar på bokföringens genomförande och strukturering,
  \item händelser i verksamheten som påverkat andra organ av sektionen, samt
  \item revisorernas uppfattning om verksamheten givet verksamhetsberättelse
    och samtal med av SM utnämnda nämndansvariga.
\end{itemize}

Revisionsberättelsen baseras på revisionsrapporten och är det dokument som
presenteras för SM vid fråga angående ansvarsfrihet. Revisionsberättelsen är en
kort sammanfattning av rapporten, med avslutande rekommendation att tillstyrka
eller avstyrka beviljande av ansvarsfrihet. Rekommendationen kan utelämnas då
särskilda skäl föreligger det emot.

\subparagraph{SM}

Vid ett SM där en revisionsberättelse skall läsas, kan revisorerna, enligt
föregående avsnitt, ge en rekommendation till SM angående beviljande av
ansvarsfrihet. SM bör beakta revisorernas samlade arbete vid efterföljande
omröstning.

Innan fråga angående ansvarsfrihet tas upp på SM skall revisorerna ansvara för
att de berörda ekonomiskt ansvariga inbjuds till SM.

\subparagraph{Normativa rekommendationer}

De rekommendationer som ges nedan bör följas för att förenkla och accelerera
revisionsförfarandet.

\subparagraph{Verksamhetsberättelse}

Det åligger sektionsordförande att ansvara för att en verksamhetsberättelse
(VB) uppförs efter (eller i samband med) avslutat verksamhetsår. Denna VB skall
(som ett minimum) innehålla en berättelse från varje ordförande för
bokföringspliktig nämnd samt bokföringspliktiga funktionärer. VB skall vara
revisorerna tillhanda innan första SM på nästkommande verksamhetsår.

\subparagraph{Bokföring}

Det åligger de ekonomiskt ansvariga i varje bokföringspliktig nämnd att lämna
en avslutad bokföring till revisorerna. Bokföringen skall vara revisorerna
tillhanda innan första SM på nästkommande verksamhetsår, om inte starka skäl
föreligger däremot.

Det åligger även de ekonomiskt ansvariga att på ett professionellt och
strukturerat sätt inventera lager och kassa vid överlämnandet till nästa
förtroendevald på posterna. Överlämningsdokumentet skall finnas revisorerna
tillhanda tillsammans med bokföringen.

\paragraph{Mandatperiod}

Revisorn väljs på Glögg SM till sakrevisor för sektionen under ett
verksamhetsår samt till funktionärsposten revisor under perioden 1 januari till
30 juni nästkommande år.

\subsubsection{Sektionshistoriker}

Sektionshistorikern skall se till att sektionens ärorika historia inte faller i glömska, dels genom att samla in historisk information och historiska föremål och dels genom att föra sagda information vidare till och visa upp sagda föremål för sektionsmedlemmarna i lämpliga sammanhang.

Sektionshistorikern ansvarar även för sektionens alumniverksamhet.

Sektionshistorikern avgör själv hur hen bäst uppfyller ändamålet. Sektionshistorikern har som
kunskapskälla tillgång till GUDAR-gruppen, Gamla Uvar på Data med Anrika Redogörelser.

Väljs på Val-SM. Har läsår som mandatperiod.

\subsubsection{Ljud- och ljusansvarig}

Väljs på Glögg-SM. Har kalenderår som mandatperiod.

\paragraph{Ändåmål}

Ljud- och ljusansvarig har till uppgift att underhålla sektionens ljud- och
ljusutrustning.

\subsubsection{Mulle/Mullerina Schmeck}

\paragraph{Ändamål}

Har till uppgift att ansvara över sektionens bil och tillhörande ekonomi.

\paragraph{Verksamhet}

Hanterar bokning, regler för bilen, reparation/underhåll, parkeringstillstånd,
försäkringar och annat som hör bilen till.

\paragraph{Organisation}

I det fall att posten är vakantsatt har Konglig Öfverdrif ansvar för bilen.

\paragraph{Mandatperiod}

Väljs på Glögg-SM. Har kalenderår som mandatperiod.

\subsubsection{Datas Art Director}

\paragraph{Ändamål}
Leder Tag Monkeys. Ansvarar tillsammans med Tag Monkeys för sektionens
grafiska utveckling och arbete.

\paragraph{Mandatperiod}

Väljs på Val-SM. Har läsår som mandatperiod.

\subsubsection{Chefredaqtör}

\paragraph{Ändamål}
Leder Redaqtionen. Ansvarar tillsammans med Redaqtionen för skriverier och
nyhetsutskick inom Informationsorganet samt för sektionstidingen dBuggen.

\paragraph{Mandatperiod}
Väljs på Glögg-SM. Har kalenderår som mandatperiod.

\subsubsection{Systemansvarig}

\paragraph{Ändamål}
Leder Crash \& Bränn. Ansvarar tillsammans med Crash \& Bränn över sektionens datasystem.

\paragraph{Mandatperiod}
Väljs på Val-SM. Har läsår som mandatperiod.

\subsubsection{Studerandeskyddsombud}

\paragraph{Ändamål}
Har till uppgift att agera som studerandeskyddsombud för sektionen. Funktionären skall såväl proaktivt som reaktivt verka för att sektionsmedlemmarnas studiemiljö är så bra som möjligt. 

\paragraph{Verksamhet}
Studerandeskyddsombudet ska:
\begin{itemize}
    \item ta emot och behandla anmälningar rörande studiemiljön för sektionsmedlemmar
    \item agera som informationskanal mellan sektionsmedlemmarna och KTH samt THS i arbetsmiljö frågor
    \item närvara på skyddsronder i lokaler där sektionsmedlemmarna ofta vistas
    \item se till att det finns en sjukvårdslåda med lämpligt innehåll i sektionslokalen
    \item regelbundet kontrollera de brandsläckare som finns i sektionslokalen
\end{itemize}

\paragraph{Organisation}
I det fall att posten är vakantsatt är Ledamot för studiesociala frågor ställföreträdande studerandeskyddsombud.

\paragraph{Mandatperiod}
Väljs på Glögg-SM. Har kalenderår som mandatperiod.

\subsubsection{Internationell Studentkoordinator}
\paragraph{Ändamål}

Internationella Studentkoordinatorn,
på engelska kallad ''International Student Coordinator'', ISC, ska tillsammans
med motsvarande post på Sektionen för Medieteknik verka för ökad integration
mellan nationella och internationella studenter på CSC-skolan.

\paragraph{Organisation}

Samtliga intresserade
sektionsmedlemmar får hjälpa till med den internationella verksamheten på sektionen.

\paragraph{Verksamhet}

ISC ska

\begin{itemize}
	\item Samordna sektionens internationella verksamhet. Detta
	inkluderar att hålla kontakten med huvudansvarig för sociala och internationella relationer
	på THS och de ansvariga för utbytesstudier på CSC:s kansli och institutionerna.
	\item Fungera som kontaktperson för internationella studenter på sektionen.
	\item Samarbeta med motsvarande post på Sektionen för Medieteknik för att verka för ökad integration mellan nationella och internationella studenter på CSC skolan.
	\item Dela information till de internationella studenterna på CSC-skolan om händelser och nämnder på datasektionen.
	\item Hålla sektionsmedlemmarna informerade om internationell verksamhet på sektionen.
	\item Främja informationsspridning på engelska inom sektionen.
\end{itemize}

\paragraph{Mandatperiod}

Väljs på Glögg-SM. Har kalenderår som mandatperiod.

\subsubsection{D-Dagenansvarig}

Ansvarar för sektionens arbetsmarknadsdag D-Dagen. Ingår i näringslivsgruppen och samarbetar med näringslivsansvarig. I händelsen att posten som Näringslivsansvarig är vakantsatt ansvarar D-Dagenansvarig även för näringslivsgruppen tills dess att posten som Näringslivsansvarig kan fyllnadsväljas.


Väljs på Glögg-SM. Har kalenderår som mandatperiod.

\section{Externa representanter}

\subsection{Representation i råd på THS}

Samtliga ordinarie ledamöter skall om möjligt delta på varje möte. Ordinarie
ledamots kontaktperson i D-rektoratet träder in som suppleant om ingen
ordinarie ledamot kan närvara, och ingen annan överenskommelse skett mellan de
bägge. I råd med öppet medlemskap får dock suppleanter naturligtvis delta på
alla möten.

\subsubsection{Näringslivsrådet}

\paragraph{Ordinarie}

Näringslivsansvarig och D-Dagenansvarig

\subsubsection{Idrottsrådet}

\paragraph{Ordinarie}

Sektionsidrottsledare

\subsubsection{Informationsrådet}

\paragraph{Ordinarie}

Informationsansvarig

\subsubsection{Internationella rådet}

\paragraph{Ordinarie}

Internationell Studentkoordinator

\subsubsection{Jämlikhetsrådet}

\paragraph{Ordinarie}

Jämlikhetsnämndens ordförande

\subsubsection{Pubmästarrådet}

\paragraph{Ordinarie}

Klubbmästaren

\subsubsection{Mottagningsrådet}

\paragraph{Ordinarie}

Mottagningens titelgrupp. Om THS styrelse så bestämmer är även
sektionsordförande ordinarie ledamot.

\paragraph{Suppleant}

Styrelseledamot för studiesociala frågor

\subsubsection{Ordföranderådet}

\paragraph{Ordinarie}

Sektionsordförande

\paragraph{Suppleant}

Vice sektionsordförande

\subsubsection{Redaktionsrådet}

\paragraph{Ordinarie}

Chefredaqtör

\paragraph{Suppleant}

Sekreterare

\subsubsection{Studiemiljörådet}

\paragraph{Ordinarie}

Styrelseledamot för studiesociala frågor

\paragraph{Suppleant}

Styrelseledamot för utbildningsfrågor

\subsubsection{Utbildningsrådet}

\paragraph{Ordinarie}

Studienämndens ordförande, Programansvarig student

\paragraph{Suppleant}

Styrelseledamot för utbildningsfrågor

\subsection{Representation inom organ på KTH}

\subsubsection{Val av representanter}

Representanter till många av dessa organ väljs inte direkt av sektionen, utan
nomineras av skolrådet till THS styrelse som sedan tillsätter posterna.

\subsubsection{Förteckning}

\paragraph{Strategiska rådet}

Fattar beslut om bland annat budget och bokslut, samt de frågor som dekanen
anser att styrelsen skall besluta om.

\paragraph{Ledningsgruppen}

Bereder och lämnar förslag inför viktigare beslut som skall fattas på skolan.

\paragraph{Arbetsmiljögruppen}

Arbetar för en bättre arbetsmiljö på skolan.

I denna grupp sitter styrelseledamot för studiesociala
frågor.

\paragraph{Grundutbildningsgruppen}

Arbetar för förbättring av grundutbildningen.

I denna grupp sitter Programansvarig student som ordinarie, samt Studienämndens
ordförande som suppleant.

\paragraph{Jämlikhet, mångfald och likabehandlingsgruppen}

Arbetar för att förbättra jämlikheten och mångfalden på skolan.

I denna grupp sitter Jämlikhetsnämndens ordförande.

\paragraph{Tjänsteförslagsnämnden}

Tjänsteförslagsnämnden har till uppgift att bereda och avge förslag beträffande
vissa anställningar.

\paragraph{Skolråd}
Skolrådet är öppet för alla studenter på Data och Media. Skolrådet väljer representanter till andra råd.

Representanter i skolrådet utses av styrelsen.

\section{Ordinarie SM}

\subsection{Förteckning}

\subsubsection{Budget-SM}

Ett SM skall hållas på hösten senast 15 november och benämnas Budget-SM.
Budget-SM skall speciellt behandla frågan om budget för nästkommande
verksamhetsår.

\subsubsection{Glögg-SM}

Ett SM skall hållas i december och benämnas Glögg-SM.

Mötesordföranden skall på Glögg-SM bära cerise tomteluva. Detta för att
försäkra sig om att ingen sektionsmedlem blir sittande i Cerise eller
motsvarande terminalinrättning på julafton.

\subsubsection{Revisions-SM}

Ett SM skall hållas på våren senast 31 mars och benämnas Revisions-SM.
Revisions-SM skall speciellt granska D-rektoratets, nämndernas och
funktionärernas berättelser samt frågan om ansvarsfrihet för D-rektoratet och
nämnder med bokföringsplikt.

\subsubsection{Val-SM}

Ett SM skall hållas efter Revisions-SM senast 15 maj och benämnas Val-SM.

\section{Förtjänsttecken och ordnar}

\subsection{Hedersdeltat}

Sektionens finaste förtjänsttecken heter Hedersdeltat och utgörs av en nål med
ett delta inramat av en eklövskrans.

\subsubsection{Syfte}

Hedersdeltat utdelas till de sektionsmedlemmar som synnerligen förtjänstfullt
verkat ideellt för sektionen.

\subsubsection{Förslagslämning}

Sektionsmedlem kan när som helst inlämna förslag på mottagare av Hedersdeltat,
med motivering, till D-rektoratet.

\subsubsection{Utdelning}

D-rektoratet utnämner mottagare av Hedersdeltat, vilka presenteras vid
Revisions-SM. Utdelning av förtjänsttecknen sker på Vårbalen eller motsvarande
högtidligt tillfälle samma år.

\subsection{Ordnar}

Sektionen har fem ordnar benämnda ``Klubbmästare Emeritus'', ``Konglig
Öfverdrif Emeritus'', ``Storasyskon Emeritus'', ``Ordförande Emeritus'' och ``Kassör Emeritus''.

\subsubsection{Ordförande Emeritus}

Ordförande Emeritus tilldelas de sektionsordförande som förtjänstfullt arbetat
under en hel mandatperiod.

Vidare gäller att Ordförande Emeriti

\begin{itemize}
  \item erhåller evigt kostnadsfritt medlemskap i sektionen som Alumnimedlem
  \item erhåller årlig speciell inbjudan till Vårbalen.
\end{itemize}

\subsubsection{Klubbmästare Emeritus}

Klubbmästare Emeritus tilldelas de Klubbmästare som förtjänstfullt arbetat
under en hel mandatperiod.

\subsubsection{Konglig Öfverdrif Emeritus}

Konglig Öfverdrif Emeritus tilldelas de Konglig Öfverdrif som förtjänstfullt
arbetat under en hel mandatperiod.

\subsubsection{Storasyskon Emeritus}

Storasyskon Emeritus tilldelas de Storasyskon som förtjänstfullt arbetat under
en hel mandatperiod.

\subsubsection{Kassör Emeritus}

Kassör Emeritus tilldelas de Kassörer som förtjänstfullt arbetat under
en hel mandatperiod.

\subsection{Funktionärsmedalj}

\subsubsection{Syfte}

Funktionärsmedaljen utdelas till de sektionsmedlemmar som förtjänstfullt under
en hel mandatperiod tjänstgjort som funktionär på sektionen.

\subsubsection{Utdelning}

Endast en medalj per funktionärspost och mandatperiod. D-rektoratet ansvarar för att medaljen utdelas på
Vårbalen eller motsvarande högtidligt tillfälle samma år.

\subsection{Projektledarmedalj}

\subsubsection{Syfte}

Projektledarmedaljen utdelas till de sektionsmedlemmar som förtjänstfullt
planerat och genomfört ett projekt vid sektionen i egenskap av projektledare
samt i förekommande fall fullständigt avslutat den ekonomiska bokföringen.

\subsubsection{Urval}

För bedömning av huruvida en projektledare arbetat förtjänstfullt ansvarar
D-rektoratet. Vid denna bedömning bör särskild vikt läggas vid att projektet
tillför något för sektionens medlemmar samt att det ekonomiska resultatet inte
med marginal understiger av sektionen godkänd budget.

\subsubsection{Utdelning}

En medalj per person och projekt utdelas. D-rektoratet ansvarar för att utdelning sker på Vårbalen eller motsvarande högtidligt tillfälle.

\section{Sektionslokalen}

Sektionslokalen kan endast bokas/hyras av sektionernas styrelser, nämnder och
funktionärer, såväl som av organ inom THS, kårföreningar och andra sektioner.
Beslut om att bevilja eller avslå bokningsbegäran fattas av
sektionslokalsansvariga från fall till fall. Bokning av sektionslokalen som
sträcker sig över lunchtid, 11:30 -- 13:30, räknas som en stängning enligt
\S\ref{sec:sektionslokal-stangning}.

\subsection{Prioritering}

Vid krockande bokningar gäller följande prioritetslista i fallande ordning.
Dock kan ingen tvingas att ändra sin bokning med mindre än två veckors varsel.
Inom en prioritetsnivå gäller först till kvarn.

\begin{enumerate}
  \item D-rektoratet/Medietekniks styrelse, sektionsnämnd eller -funktionär
  \item THS-organ
  \item Kårförening eller annan sektion vid THS.
\end{enumerate}

\subsection{Alkohol}

Det är inte tillåtet att medföra egen alkohol till lokalen, vare sig för
försäljning eller enskilt bruk. Om alkoholservering önskas måste en behörig
ansvarig från sektionerna närvara. Sektionen förbehåller sig alltid rätten att
neka alkoholservering.

\subsection{Övriga regler}

Det får max befinna sig 150 personer i lokalen. Ingen sektionsmedlem kan vägras
tillträde till lokalen, även om den är bokad. Dock skall medlemmar alltid visa
största möjliga hänsyn mot den/de som bokat lokalen. Den person som hyr
sektionslokalen är personligen ansvarig för de aktiviteter som förekommer där
under uthyrningen. I övrigt gäller KTH-handbokens regler för fester och
sammankomster i KTH:s lokaler.

\subsection{Undantag}

Sektionslokalsansvariga kan besluta om undantag från dessa regler, i den mån
det är förenligt med THS och KTH:s regler samt svensk lagstiftning, om
särskilda skäl föreligger.

\subsection{Stängning}
\label{sec:sektionslokal-stangning}

Sektionslokalen får stängas under en bestämd tidsperiod efter likalydande beslut av respektive
sektionsstyrelse. Stängning måste dock annonseras i sektionernas officiella informationskanaler senast 4 läsdagar i förväg och beslut om stängning skall alltid prövas på nästföljande styrelsemöte.


\subsection{Förbud}

\subsubsection{Nötförbud}
Cashewnötter, hasselnötter, jordnötter, makadamianötter, mandlar, paranötter, pekannöt-
ter, pistagenötter och valnötter är ej tillåtna i sektionslokalen. Produkter som kan innehålla
spår av nötter är tillåtna.

\subsubsection{E-cigarett-förbud}
E-cigarett är helt förbjudet i sektionslokalen.

\subsection{Samboendeavtal}
Samboendeavtalet beskriver vilka regler som gäller för sektionslokalen samt förtydligar
hur verksamheten i lokalen ska hanteras. Avtalet skrivs för att stärka samarbetet mellan
Konglig Datasektionen och Sektionen för Medieteknik. Samboendeavtalet revideras
årligen av respektive sektions styrelser i samråd med de sektionslokalsansvariga och
med synpunkter från de båda sektionernas engagerade.
\section{Övrigt}

\subsection{Visdomsord}

Det var bättre förr.

\subsubsection{Mer visdomsord}

Och ju förr desto bättre.

\end{document}

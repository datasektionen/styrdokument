\documentclass{dgovdoc}

\usepackage[swedish]{babel}
\usepackage[T1]{fontenc}

\usepackage{hyperref}

\title{Stadgar}

\begin{document}

\maketitle

\section{Allmänt}

\subsection{Namn}
\label{sec:namn}

Föreningens namn är Konglig Datasektionen, nedan benämnd sektionen.

\subsection{Ändamål}

Sektionen har till ändamål att främja sina medlemmars studier och vad som har
sammanhang med dessa.

Sektionens uppgifter är

\begin{itemize}
  \item att utveckla och upprätthålla kamratskap och sammanhållning bland
    sektionens medlemmar
  \item att skapa och upprätthålla goda kontakter med närstående personer och
    organisationer
  \item att aktivt motverka diskriminering inom sektionen.
\end{itemize}

\subsection{Säte}

Sektionen har sitt säte i Stockholm.

\subsection{Kårtillhörighet}

Sektionen tillhör Tekniska Högskolans Studentkår, THS.

\subsection{Verksamhetsår}

Sektionens verksamhetsår löper från 1 januari till 31 december.

\subsection{Styrdokument}
\label{sec:styrdokument}

\subsubsection{Tillgänglighet}

Gällande stadgar och andra styrdokument skall finnas tillgängliga för samtliga
sektionsmedlemmar på sektionens officiella webbplats.

\subsubsection{Tolkning}

Skulle tveksamhet uppstå angående dessa stadgar, tolkas dessa av SM. Mellan SM
tolkas stadgarna, i stigande företrädesordning, av D-rektoratet och
revisorerna. Sådan tolkning skall dock alltid prövas på nästföljande SM.

\subsubsection{Konflikt}

Vid konflikt mellan dessa stadgar eller andra av sektionens styrdokument och
THS stadgar eller andra styrdokument skall THS föreskrifter äga företräde. Om
sådan konflikt uppmärksammas åligger det D-rektoratet att snarast inkomma till
SM med proposition om nödvändiga ändringar.

\subsubsection{Stadgar}

\paragraph{Stadgeändring}

Stadgarna ändras genom likalydande beslut på två på varandra följande SM, varav
minst det ena måste vara ett ordinarie enligt \S\ref{sec:ordinarie_sm}.
D-rektoratet skall föra upp fråga om andra läsning av vilande stadgeändring på
kallelse och föredragningslista till nästa SM.

\paragraph{Dispens från stadgarna}

Dispens från bestämmelse i dessa stadgar kan beviljas om SM enhälligt beslutar
så. Dispens får dock ej medges från \S\ref{sec:namn} -- \S\ref{sec:styrdokument}
och ej heller om det är till nackdel för enskild sektionsmedlem. Beslut om
dispens skall motiveras i mötesprotokollet.

\subsubsection{Reglemente}

\paragraph{Ändamål}

Reglementet reglerar sektionsverksamheten där stadgarna ej är tillräckligt
utförliga. Reglementet är alltid underordnat stadgarna.

\paragraph{Reglementesändring}

Reglementet kan ändras genom beslut på ett SM. Punkter som måste finnas enligt
stadgarna kan dock inte avskaffas utan stadgeändring.

\subsection{Beslutsnivåer}

Beslut fattas med enkel majoritet om inget annat är föreskrivet.

\subsection{Officiella informationskanaler}
\label{sec:officiella_informationskanaler}

Information som anslås skall sättas upp i minst A4-format på väl synlig plats i
sektionslokalen, samt göras tillgänglig för samtliga sektionsmedlemmar på
sektionens officiella webbplats.

\subsection{Definitioner}

\subsubsection{Läsdag}

Definitionen av läsdagar som benämns i sektionens styrdokument är
måndag till och med fredag under terminstid, inklusive tentamensperiod, dock ej helgdagar.

\subsection{Firmatecknare}

Ordförande och kassör tecknar firman var för sig. D-rektoratet kan fatta beslut
om ytterligare firmatecknare. Ett sådant beslut ska meddelas på nästa
sektionsmöte.

\section{Medlemskap}

Sektionsmedlem är

\begin{itemize}
  \item ordinarie sektionsmedlem enligt \S\ref{sec:ordinarie_sektionsmedlem}
  \item hedersmedlem enligt \S\ref{sec:hedersmedlem}
  \item alumnimedlem enligt \S\ref{sec:alumnimedlem}
  \item juniormedlem enligt \S\ref{sec:juniormedlem}
\end{itemize}

\subsection{Ordinarie sektionsmedlem}
\label{sec:ordinarie_sektionsmedlem}

Ordinarie sektionsmedlem är medlem i THS som enligt THS föreskrifter eller
beslut tillhör sektionen.

\subsubsection{Rättigheter}

Ordinarie sektionsmedlem har rätt att

\begin{itemize}
  \item deltaga med yttranderätt och rösträtt på SM
  \item få motion eller interpellation behandlad av SM
  \item kandidera till samtliga förtroendeuppdrag inom sektionen
  \item närvara på DM såvida det inte beslutats om lyckta dörrar.
\end{itemize}

\subsection{Hedersmedlem}
\label{sec:hedersmedlem}

Sektionen kan utse till hedersmedlem sådan person som synnerligen främjat
sektionens intressen och strävanden. Förslag till hedersmedlem lämnas av
sektionsmedlem. Hedersmedlem utses av SM med 5/6 majoritet. Faller fråga om val
av hedersmedlem införs varken förslag eller beslut i protokoll.

\subsection{Alumnimedlem}
\label{sec:alumnimedlem}

Alumnimedlem är medlem som har anknytning till sektionen och antingen under
samma period är stödmedlem hos THS eller av annan anledning är alumnimedlem i
sektionen.

\subsubsection{Rättigheter}

Alumnimedlem har rätt att:

\begin{itemize}
\item deltaga med närvaro-, yttrande- och yrkanderätt på SM.
\item kandidera till samtliga förtroendeuppdrag inom sektionen.
\item närvara på DM såvida det inte beslutats om lyckta dörrar.
\end{itemize}

\subsection{Juniormedlem}
\label{sec:juniormedlem}

Endast för nyligen antagen till Datateknikprogrammet. Medlemskapet gäller från
uppropet och två månader framåt. Juniormedlem har närvarorätt och yttranderätt
på SM.

\subsection{Medlemsavgift}

Medlemsavgift för ordinarie medlemmar och alumnimedlemmar fastställs av THS
centralt. Hedersmedlemmar är befriade från medlemsavgift.

\section{Sektionsmötet}

\subsection{Ändamål}

Sektionsmötet, SM, är sektionens högsta beslutande organ.

\subsection{Sammansättning}

Vid SM har samtliga ordinarie sektionsmedlemmar närvarorätt, yttranderätt,
yrkanderätt och rösträtt. Sektionens revisorer enligt \S\ref{sec:revisorer} har
närvaro-, yttrande- och yrkanderätt. Hedersmedlemmar, alumnimedlemmar och
ledamöter i THS styrelse har närvaro- och yttranderätt. Dessutom kan SM
adjungera utomstående med närvarorätt och eventuellt även yttranderätt.

\subsection{Uppgifter}

Det åligger SM

\begin{itemize}
  \item att fastställa riktlinjer och budget för sektionens verksamhet
  \item att granska funktionärers och revisorers berättelser samt sektionens
    ekonomiska redovisning
  \item att ta ställning till ansvarsfrihet för D-rektoratet
  \item att om sektionsmedlem så önskar granska protokoll från DM
  \item att välja funktionärer, med undantag av de som väljs vid urnval
  \item att genomföra fyllnadsval vid behov, även till poster som vanligen
    väljs vid urnval
\end{itemize}

\subsection{Kallelse}
\label{sec:kallelse}

D-rektoratet kallar till ordinarie och extra SM.

Kallelse till ordinarie SM skall anslås enligt
\S\ref{sec:officiella_informationskanaler} samt tillsändas THS styrelse och THS
sakrevisorer senast 15 läsdagar före mötet för att mötet skall anses vara
behörigt utlyst.

Kallelse till extra SM skall anslås enligt
\S\ref{sec:officiella_informationskanaler} samt tillsändas THS styrelse och THS
sakrevisorer senast 8 läsdagar före mötet för att mötet skall anses vara
behörigt utlyst.

Föredragningslista och övriga handlingar skall anslås jämte kallelse senast
5 läsdagar före mötet.

Om minst 30 sektionsmedlemmar, sektionsrevisor enligt \S\ref{sec:revisorer} eller
THS styrelse så begär hos D-rektoratet, skall extra SM hållas inom 20 läsdagar.

Kallelse skall innehålla information om en reservtid och lokal där mötet
återupptas om ajournering beslutas enligt \S\ref{sec:ajournering}. Reservtiden
skall vara senast fem (5) läsdagar efter den ordinarie tiden för mötet.

\subsection{Beslutsmässighet}

SM är beslutsmässigt när mötet är behörigt utlyst enligt 
\S\ref{sec:kallelse} och minst 10 sektionsmedlemmar är närvarande.

\subsection{Beslut}

Beslut kan endast fattas i fråga som antingen enligt stadgarna skall behandlas
eller berörs av proposition eller motion. Beslut fattas med enkel majoritet
såvida inget annat stadgats. Vid lika röstetal har mötesordföranden
utslagsröst, förutom vid personval då lotten avgör. Sluten omröstning skall ske
om någon röstberättigad deltagare så begär. SM fattar beslut med parlamentarisk
voteringsmetod. Sektionsmedlem äger inte rätt att rösta genom ombud, utan
endast genom personlig närvaro på SM.

Ingen får delta i beslut eller leda sammanträdet när frågan om ansvarsfrihet
för honom/henne själv behandlas.

\subsubsection{Reservation}

Varje röstberättigad deltagare på SM kan reservera sig mot fattat beslut.
Reservation anmäls i samband med beslutet, och lämnas in skriftligen till
mötessekreteraren senast 24 timmar efter mötets avslutande. Reservationer skall
föras in i protokollet.

\subsection{Protokoll}

Vid SM skall diskussionsprotokoll föras av mötessekreterare och justeras av
mötesordföranden jämte två av mötet utsedda justerare. Protokoll skall
innehålla en förteckning över närvarande, röstberättigade medlemmar. Protokoll
skall i justerat skick anslås enligt \S\ref{sec:officiella_informationskanaler}
samt tillsändas THS styrelse inom 14 dagar.

\subsection{Interpellation, motion och proposition}

Motion eller interpellation till SM skall vara D-rektoratet tillhanda senast 10
läsdagar före SM. D-rektoratet skall skriftligen besvara samtliga motioner.

Förslag från D-rektoratet benämns proposition.

Interpellation skall besvaras skriftligen av den funktionär den ställts till,
eller av D-rektoratet om den ställts till dem.

D-rektoratet ansvarar för att motioner, interpellationer, svar på dessa samt
propositioner anslås tillsammans med föredragningslistan.

\subsection{Sammanträden}

Det skall förflyta minst fem läsdagar mellan två på varandra följande SM. SM
får inte hållas under tentamensperiod eller ferie.

\subsubsection{Ordinarie SM}
\label{sec:ordinarie_sm}

Som ordinarie SM räknas de SM som regleras i reglementet, samt övriga, av
D-rektoratet, utlysta ordinarie SM. Som ordinarie SM räknas ej extra SM. Det
skall hållas minst ett ordinarie SM per termin.

\subsubsection{Extra SM}

D-rektoratet kan, själv eller på anmodan, kalla till extra SM. Extra SM kan
endast behandla den eller de frågor som angivits i kallelsen, således behandlas
ej övriga motioner, interpellationer eller propositioner. Dock kan övrig fråga
väckas.

\subsection{Ajournering}
\label{sec:ajournering}

Mötets ordförande äger rätt att ajournera mötet till den, enligt 
\S\ref{sec:kallelse} i kallelsen angivna reservtiden.

\section{D-rektoratet}

\subsection{Ändamål}

D-rektoratet är sektionens styrelse och högsta verkställande organ.

\subsection{Sammansättning}

D-rektoratet består av

\begin{itemize}
  \item Sektionsordförande
  \item Vice Sektionsordförande
  \item Kassör
  \item Sekreterare
  \item Ledamot för sociala frågor och relationer
  \item Ledamot för studiemiljöfrågor
  \item Ledamot för utbildningsfrågor
\end{itemize}

Dessa har närvaro-, yttrande-, yrkande- och rösträtt vid DM. Sektionens
revisorer enligt har närvaro-, yttrande- och yrkanderätt vid DM. Funktionärer
har närvaro- och yttranderätt vid DM. Övriga sektionsmedlemmar har närvarorätt
vid DM. Därutöver äger D-rektoratet rätt att adjungera person med närvaro-
eller närvaro- och yttranderätt för viss fråga eller helt möte. D-rektoratet
äger vidare, om synnerliga skäl föreligger, rätt att besluta om lyckta dörrar,
vilket utestänger samtliga utan yrkanderätt.

\subsection{D-rektoratsmöte}
\label{sec:d_rektoratsmote}

\subsubsection{Kallelse}

Sektionsordförande kallar till D-rektoratsmöte, DM. Kallelsen skall anslås
enligt \S\ref{sec:officiella_informationskanaler} samt skickas med e-post till
D-rektoratets ledamöter och sektionens funktionärer senast 5 läsdagar före
mötet.

\subsubsection{Beslut}

DM är beslutsmässigt om minst hälften av dess ledamöter är närvarande, och
mötet är behörigt utlyst enligt \S\ref{sec:d_rektoratsmote} Vid lika röstetal har
mötesordförande utslagsröst.

\subsubsection{Protokoll}

På DM skall protokoll föras. Protokollet skall justeras av mötesordföranden
jämte en av mötet utsedd justerare. Protokollet skall anslås enligt
\S\ref{sec:officiella_informationskanaler} i justerat skick senast 14 dagar efter
mötet.

\subsection{Uppgifter}

Det åligger D-rektoratet

\begin{itemize}
  \item att sköta sektionens löpande förvaltning
  \item att verkställa av SM fattade beslut
  \item att i brådskande fall utöva SM:s befogenheter. Sådant fall skall dock
    alltid prövas på nästkommande SM
  \item att efter skriftlig begäran från en av SM vald funktionär entlediga
    densamme
  \item att vid behov och efter majoritetsbeslut vid DM tillförordna
    intresserad sektionsmedlem till vakant post inom sektionen. D-rektoratet
    får dock ej tillförordna D-rektoratsledamot, sektionsrevisor, ledamot eller
    suppleant i kårfullmäktige eller valberedare
  \item att vid behov och efter majoritetsbeslut vid DM utöva ordförandeskap
    för nämnd i dess ordförandes ställe.
  \item att svara för att verksamhetsplan, budget, verksamhetsberättelse och
    årsbokslut upprättas
  \item att, om så anses nödvändigt, avsätta en av sektionen vald funktionär,
    dock ej styrelseledamot, revisor, kårfullmäktigesuppleant,
    kårfullmäktigeledamot eller valberedare. En sådan avsättning skall dock
    alltid prövas på nästkommande SM.
\end{itemize}

\subsection{Brådskande ärenden}

I brådskande fall äger D-rektoratet rätt att utöva SM:s befogenheter.
D-rektoratet äger dock ej därigenom rätt att ändra stadgar eller reglemente
eller att bevilja dispens från stadgarna. Beslut enligt detta stycke skall
prövas på nästföljande SM.

I brådskande fall äger sektionsordförande rätt att utöva D-rektoratets
befogenheter. Sektionsordförande äger dock ej därigenom rätt att utöva SM:s
befogenheter enligt första stycket eller att besluta om utgifter överstigande
2000 SEK. Beslut enligt detta stycke skall prövas på nästföljande DM.

\subsection{Beslutsrätt gällande mindre summor}

D-rektoratet äger rätt ta beslut gällande summor under 10 000 SEK vilket belastar budgetposten Styrelsens dispositionsfond.

\subsection{Ställföreträdande sektionsordförande}

Om sektionsordförande är oförmögen att göra så, utövar vice sektionsordförande
dennes befogenheter, och fullgör dennes plikter.

\subsection{Per capsulam-beslut}

Vid per capsulam beslut gäller 2/3-majoritet och att beslut prövas på
nästkommande DM.

\subsection{D-rektiv}

D-rektoratet må, om det så önskar, utfärda D-rektiv, vilka utgöra
rekommendationer å de enskilda sektionsmedlemmarnas liv och leverne.

\section{Organisation}

\subsection{Nämnder}
\label{sec:namnder}

\subsubsection{Ändamål}

En nämnd är ett officiellt sektionsorgan med syfte att ansvara för en viss del
av sektionens verksamhet. Nämnder driver sin verksamhet självständigt inom
ramen för av SM och D-rektoratet fattade beslut. Nämnder är de, som upptas i
reglementet.

\subsubsection{Sammansättning och verksamhet}

En nämnds sammansättning och verksamhet regleras i reglementet.

\paragraph{Ordförande}

För varje nämnd skall det finnas en eller flera funktionärer som är ordförande.
Nämndens ordförande är ansvarig för nämndens verksamhet samt att dess
reglemente hålls aktuellt.

\subsubsection{Skyldigheter}

Nämnd är skyldig att upprätta verksamhetsberättelse, samt även annars på
anmodan från D-rektoratet eller SM fullständigt redovisa sin verksamhet för
densamme.

\subsubsection{Valberedning}

Valberedningen skall enligt \S\ref{sec:officiella_informationskanaler} anslå en
nomineringslista senast 15 läsdagar före SM då ordinarie val sker. På denna
lista kan sektionsmedlemmar nominera funktionärer.

Nominering till funktionärspost måste lämnas in senast 5 läsdagar före det SM
där valet sker. Nominering till funktionärspost måste accepteras senast en
läsdag före det SM där valet sker för att kandidaturen ska vara giltig.

Valberedningen skall tillfråga de nominerade och i anslutning till handlingarna
för SM då val sker anslå en ny lista på samtliga nominerade som tackat ja till
nomineringen.

\paragraph{Urnval}

Sektionsordförande, vice sektionsordförande och övriga ledamöter och
suppleanter till THS Kårfullmäktige väljs med urnval i enlighet med sektionens
reglemente och THS stadgar.

\subsubsection{Obligatoriska nämnder}

Det skall finnas en valberedning, en studienämnd, en sektionslokalnämnd samt en
mottagningsnämnd.

\subsection{Funktionärer}

\subsubsection{Ändamål}

Funktionär är den som av SM eller vid urnval har valts till ett
förtroendeuppdrag. En funktionärs verksamhet och uppdrag regleras i
reglementet.

\subsubsection{Skyldigheter}

Funktionär ansvarar för sitt verksamhetsområde samt för att funktionärens del
av reglementet hålls aktuellt. Funktionär är skyldig att löpande hålla
D-rektoratet informerat om sitt verksamhetsområde, samt att på anmodan från
D-rektoratet eller SM fullständigt redovisa sin verksamhet för densamme.

\subsubsection{Mandatperiod}

Funktionärs mandatperiod sammanfaller med verksamhetsår om inget annat är
föreskrivet i reglementet. Ordinarie val skall hållas på mandatperiodens sista
ordinarie SM.

\subsubsection{Obligatoriska funktionärer}

Utöver D-rektoratets ledamöter, revisorer och ordförande för de under
\S\ref{sec:namnder} uppräknade nämnderna, som regleras särskilt, skall det
finnas en programansvarig student och ett studerandeskyddsombud.

\subsection{Projekt}

\subsubsection{Ändamål}

Ett projekt är ett tillfälligt sektionsorgan med syfte att genomföra för
projektet avsatt ändamål. Projekt driver sin verksamhet självständigt inom
ramen för av SM och D-rektoratet fattade beslut. Projekt startas genom
SM-beslut och avslutas då verksamheten är genomförd och eventuell bokföring
avklarad.

\subsubsection{Projektledare}

För varje projekt skall det finnas en eller flera personer som är
projektledare. Som projektledare kan bara räknas sektionsmedlem som tillsatts
genom beslut av SM eller D-rektoratet. Projektledare kan formellt ha en annan
titel såsom direqteur, marskalk, general eller liknande om denna titel har
godkänts av SM eller D-rektoratet.

\subsubsubsection{Skyldigheter}

Projektledare är ansvarig för projektets verksamhet, ekonomi, bokföring samt
val av projektmedlemmar om inte annat beslutas av SM. Projektledare är
skyldig att på anmodan från D-rektoratet eller SM fullständigt redovisa
projektets verksamhet och ekonomi för densamme.

\subsubsubsection{Rättigheter}

D-rektoratet ansvarar för att projektledare i aktiva projekt får ta del av
samma information som sektionens funktionärer samt att även dessa bjuds
in till funktionärsmiddagar och liknande tillställningar. D-rektoratet äger
rätt att fritt bedöma vilka projekt som anses vara aktiva.

\section{Revision}

\subsection{Revisorer}
\label{sec:revisorer}

SM skall utse två revisorer. Tillsammans med av THS därtill utsedda revisorer
utgör dessa sektionens revisorer. Om inte THS beslutar annorlunda skall
revisionsberättelsen undertecknas av minst två av dessa.

\subsubsection{Befogenheter}

Revisorerna har rätt

\begin{itemize}
  \item att närhelst de så önskar ta del av samtliga räkenskaper, protokoll och
    andra handlingar
  \item att begära och erhålla upplysningar rörande verksamhet och förvaltning
  \item att bevaka samtliga sektionsorgans sammanträden med yttrande och
    yrkanderätt
  \item att kalla till möte med samtliga sektionsorgan.
\end{itemize}

\subsubsection{Uppgifter}

Det åligger revisorerna

\begin{itemize}
  \item att fortlöpande granska sektionens förvaltning och verksamhet
  \item att senast 5 läsdagar före de SM vid vilka fråga om ansvarsfrihet
    behandlas anslå revisionberättelse enligt
    \S\ref{sec:officiella_informationskanaler} samt inlämna revisionberättelse
    till D-rektoratet.
\end{itemize}

\subsection{Verksamhetsberättelse och årsbokslut}

Sektionens verksamhetsberättelse och årsbokslut skall överlämnas till
revisorerna senast 15 läsdagar före det SM på vilka de skall granskas, samt
anslås enligt \S\ref{sec:officiella_informationskanaler} senast 10 läsdagar före
samma SM.

\end{document}
